\documentclass[conference,compsoc,onecolumn]{IEEEtran}
\IEEEoverridecommandlockouts
% The preceding line is only needed to identify funding in the first footnote. If that is unneeded, please comment it out.
\usepackage{cite}
\usepackage{amsmath,amssymb,amsfonts}
\usepackage{algorithmic}
\usepackage{graphicx}
\usepackage{textcomp}
\usepackage{xcolor}
\usepackage[spanish,activeacute]{babel}
\def\BibTeX{{\rm B\kern-.05em{\sc i\kern-.025em b}\kern-.08em
    T\kern-.1667em\lower.7ex\hbox{E}\kern-.125emX}}
\begin{document}

\title{Proyecto Google Play Store Apps}\\


\author{\IEEEauthorblockN{1\textsuperscript{st} Esneider Pantoja}
\IEEEauthorblockA{\textit{Facultad de ciencias naturales}\\ { e ingenieria} \\
\textit{Universidad De Bogota}\\ {Jorge Tadeo lozano}\\
Bogota D.C \\
esneider.pantojad@utadeo.edu.co}
\and
\IEEEauthorblockN{2\textsuperscript{nd} Juan Naranjo}
\IEEEauthorblockA{\textit{Facultad de ciencias naturales}\\{ e ingenieria} \\
\textit{Universidad De Bogota}\\ {Jorge Tadeo lozano}\\
Bogota D.C \\
juanf.naranjog@utadeo.edu.co}
\and
\IEEEauthorblockN{3\textsuperscript{rd} Cristian Hernández}
\IEEEauthorblockA{\textit{Facultad de ciencias naturales}\\{ e ingenieria} \\
\textit{Universidad De Bogota}\\ {Jorge Tadeo lozano}\\
Bogota D.C \\
cristianc.hernandezs@utadeo.edu.co}

}

\maketitle

\begin{abstract}
En el presente proyecto se contempla el proceso de analisis datos, basados en Google Play Store App.Una base de datos en la cual nos entregan 10 mil datos acerca de apps que ofrese play store.
partiendo de una depuracion y organizacion de datos utilizando las herramientas de pyton como pandas, se logra una mejor visualizacionde los resultados que se presentan en el siguiente reporte 
\end{abstract}


\section{Introduccion}
Si bien muchos conjuntos de datos públicos (en Kaggle y similares) proporcionan datos de la App Store de Apple, no hay muchos conjuntos de datos equivalentes disponibles para las aplicaciones de Google Play Store en la web. Al profundizar, descubrí que la página de la App Store de iTunes implementa una estructura similar a un apéndice bien indexada para permitir un raspado web simple y fácil. Por otro lado, Google Play Store utiliza sofisticadas técnicas modernas (como la carga dinámica de páginas) utilizando JQuery, lo que hace que el scraping sea más desafiante. 

\section{Método de visualización}

Partiendo de la necesidad de tener resultados coherente y acordes a las necesidades del entorno se trabajo mediante la metodologia CRISP-DM  (Cross Industry Standard Process for Data Mining) para evaluar la trazabilidad de los datos entregados por la pagina kaggle: Google play store. Para ello se aplicaron los 6 pasos guia de dicha metodologia (comprensión del problema, comprensión de datos, preparación de datos, modelado, evaluación del modelo e implementación del mismo).

Con un proceso inicial de limpieza, transformacion y revision de los datos usamos las herramientas entregadas en clase para tal fin.Con PYTHON como herramienta principal se uso (NUMPY) con sus extencion (PANDAS). para lograr un tratamiento de los datos entregados y asi obtener una visualizacion coherente y optima de los datos utilizables y operacionales para un analisis postumo, logrando una toma de desiciones acertiva.


\section{Análisis de los datos}

\subsection {Comprensión del problema:}

Se hizo entrega de una base de datos para nuestro archivo nos indica un total de 10841 registros dividos en 13 columnas \"Caracteristicas\" diferentes para un total de 140933 entradas.Es evidente la necesidad de aplicacion de tecnicas de Machine learning,para la depuracion de datos obsoletos o que esten fuera de contexto para las necesidades de nuestro proyecto.

Nuestro objetivo es Identificar el tipo de applicaciones mas demandas dentro de la plataforma Google Play.bajo delimitaciones especificas de rangos de publico,como rango de edades de los ususrios, zonas geograficas de mayor demanda y aplicaciones de preferencia.\\

\subsection {Comprensión de datos:}

Nos dirigimos a la pagina kaggle: Google play store. Descargamos el conjunto de datos referentes al proyecto en la opcion \"Download (2 MB)\" la cual contiene los dataset del proyecto.
Conjunto de datos:
-googleplaystore.csv (1.3 MB) Este archivo contiene los detalles de las aplicaciones en Google Play. con 13 características que describen una aplicación dada.
-googleplaystore_user_reviews.csv (7.31 MB) Este archivo contiene las primeras 100 reseñas \"más relevantes\" de cada aplicación. Cada texto/comentario de revisión ha sido preprocesado y atribuido con 3 características: Sentimiento, Polaridad del sentimiento y Subjetividad del sentimiento.
-license.txt (165 B) Este trabajo tiene la licencia Creative Commons Attribution 3.0 Unported License. Para ver una copia de esta licencia, visite creativecommons.\\
{
\subsection {Preparación de datos:}

Con la implementacion de Python lo primero en hacerse fue la importacion de las librerias:
numpy es una libreria que da soporte para crear vectores y matrices grandes multidimensionales, junto con una gran colección de funciones matemáticas de alto nivel para operar con ellas.
pandas Es una liberia como extensión de NumPy para manipulación y análisis de datos, en particular ofrece estructuras de datos y operaciones para manipular tablas numéricas y series temporales.

-Para examinar el contenido del archivo ejecutamos la funcion head() que nos indica los 5 primeros registros del dataSet.

\begin{figure} [h]
  \centering
   \includegraphics[width=1\linewidth]{tabla1.png}
  \caption{“5 primeros registros del dataSet.” \cite{alajo2013aplicacion}}
  \label{partes }
\end{figure}

}
\\
-Para saber todas las columnas de nuestro archivo ejecutamos el siguiente atributo de dataset >> colums esto nos permitira conocer todos las caracteristicas de nuestro archivo.

\begin{figure} [h]
  \centering
   \includegraphics[width=0.5\linewidth]{tipo de datos.png}
  \caption{“columnas del archivo.” \cite{alajo2013aplicacion}}
  \label{partes }
\end{figure}

-Para conocer el tipo de datos de las columnas ejecutamos info().

\begin{figure} [H]
  \centering
   \includegraphics[width=0.4\linewidth]{tipo de datos 2.png}
  \caption{“tipos de datos.” \cite{alajo2013aplicacion}}
  \label{partes }
\end{figure}

-Agrupamos nuestro conjunto de datos por \"Category\".
Una vez agrupado nuestro conjunto de datos, generamos los datos estadisticos para aquellas categorias donde se pueda aplicar, esto se logra con describe()
Con include = \'all\' obtenemos el conjunto estadistico de todas las columnas
np.number incluye solo las columnas numericas del DataFrame\\

% Table generated by Excel2LaTeX from sheet 'Hoja1'
\begin{tabular}{|r|r|r|r|r|r|r|r|r|}
\hline
           &                                                                         \multicolumn{ 8}{|c|}{Rating} \\
\hline
           &      count &       mean &        std &        min &       25\% &       50\% &       75\% &        max \\
\hline
  Category &            &            &            &            &            &            &            &            \\
\hline
ART_AND_DESIGN &       62.0 &  4.358.065 &   0.358297 &        3.2 &      4.100 &        4.4 &      4.700 &        5.0 \\
\hline
AUTO_AND_VEHICLES &       73.0 &  4.190.411 &   0.543692 &        2.1 &      4.000 &        4.3 &      4.600 &        4.9 \\
\hline
    BEAUTY &       42.0 &  4.278.571 &   0.362603 &        3.1 &      4.000 &        4.3 &      4.575 &        4.9 \\
\hline
BOOKS_AND_REFERENCE &      178.0 &  4.346.067 &   0.429046 &        2.7 &      4.100 &        4.5 &      4.600 &        5.0 \\
\hline
  BUSINESS &      303.0 &  4.121.452 &   0.624422 &        1.0 &      3.900 &        4.3 &      4.500 &        5.0 \\
\hline
    COMICS &       58.0 &  4.155.172 &   0.537758 &        2.8 &      3.825 &        4.4 &      4.500 &        5.0 \\
\hline
COMMUNICATION &      328.0 &  4.158.537 &   0.426192 &        1.0 &      4.000 &        4.3 &      4.400 &        5.0 \\
\hline
    DATING &      195.0 &  3.970.769 &   0.630510 &        1.0 &      3.700 &        4.1 &      4.400 &        5.0 \\
\hline
 EDUCATION &      155.0 &  4.389.032 &   0.251894 &        3.5 &      4.200 &        4.4 &      4.600 &        4.9 \\
\hline
ENTERTAINMENT &      149.0 &  4.126.174 &   0.302556 &        3.0 &      3.900 &        4.2 &      4.300 &        4.7 \\
\hline
    EVENTS &       45.0 &  4.435.556 &   0.419499 &        2.9 &      4.200 &        4.5 &      4.700 &        5.0 \\
\hline
    FAMILY &     1747.0 &  4.192.272 &   0.508026 &        1.0 &      4.000 &        4.3 &      4.500 &        5.0 \\
\hline
   FINANCE &      323.0 &  4.131.889 &   0.642108 &        1.0 &      4.000 &        4.3 &      4.500 &        5.0 \\
\hline
FOOD_AND_DRINK &      109.0 &  4.166.972 &   0.548070 &        1.7 &      4.000 &        4.3 &      4.500 &        5.0 \\
\hline
      GAME &     1097.0 &  4.286.326 &   0.365375 &        1.0 &      4.100 &        4.4 &      4.500 &        5.0 \\
\hline
HEALTH_AND_FITNESS &      297.0 &  4.277.104 &   0.617822 &        1.4 &      4.100 &        4.5 &      4.600 &        5.0 \\
\hline
HOUSE_AND_HOME &       76.0 &  4.197.368 &   0.368411 &        2.8 &      4.000 &        4.3 &      4.500 &        4.8 \\
\hline
LIBRARIES_AND_DEMO &       65.0 &  4.178.462 &   0.378522 &        3.1 &      4.000 &        4.2 &      4.400 &        5.0 \\
\hline
 LIFESTYLE &      314.0 &  4.094.904 &   0.693907 &        1.5 &      3.800 &        4.2 &      4.600 &        5.0 \\
\hline
MAPS_AND_NAVIGATION &      124.0 &  4.051.613 &   0.519926 &        1.9 &      3.775 &        4.2 &      4.400 &        4.9 \\
\hline
   MEDICAL &      350.0 &  4.189.143 &   0.663581 &        1.0 &      4.000 &        4.3 &      4.600 &        5.0 \\
\hline
NEWS_AND_MAGAZINES &      233.0 &  4.132.189 &   0.536707 &        1.7 &      3.900 &        4.2 &      4.500 &        5.0 \\
\hline
 PARENTING &       50.0 &  4.300.000 &   0.517845 &        2.0 &      4.100 &        4.4 &      4.675 &        5.0 \\
\hline
PERSONALIZATION &      314.0 &  4.335.987 &   0.352732 &        2.5 &      4.200 &        4.4 &      4.600 &        5.0 \\
\hline
PHOTOGRAPHY &      317.0 &  4.192.114 &   0.462896 &        2.0 &      4.000 &        4.3 &      4.500 &        5.0 \\
\hline
PRODUCTIVITY &      351.0 &  4.211.396 &   0.504931 &        1.0 &      4.100 &        4.3 &      4.500 &        5.0 \\
\hline
  SHOPPING &      238.0 &  4.259.664 &   0.404577 &        1.6 &      4.100 &        4.3 &      4.500 &        5.0 \\
\hline
    SOCIAL &      259.0 &  4.255.598 &   0.413809 &        1.9 &      4.100 &        4.3 &      4.500 &        5.0 \\
\hline
    SPORTS &      319.0 &  4.223.511 &   0.427857 &        1.5 &      4.100 &        4.3 &      4.500 &        5.0 \\
\hline
     TOOLS &      734.0 &  4.047.411 &   0.616143 &        1.0 &      3.800 &        4.2 &      4.400 &        5.0 \\
\hline
TRAVEL_AND_LOCAL &      226.0 &  4.109.292 &   0.504691 &        2.2 &      3.900 &        4.3 &      4.400 &        5.0 \\
\hline
VIDEO_PLAYERS &      160.0 &  4.063.750 &   0.551098 &        1.8 &      3.800 &        4.2 &      4.400 &        4.9 \\
\hline
   WEATHER &       75.0 &  4.244.000 &   0.331353 &        3.3 &      4.050 &        4.3 &      4.500 &        4.8 \\
\hline
\end{tabular}\\  

-Visualizacion de los datos

\begin{figure} [H]
  \centering
   \includegraphics[width=0.8\linewidth]{grafico1.png}
  \caption{“numero de calificaciones por categoria.” \cite{alajo2013aplicacion}}
  \label{partes }
\end{figure}

- aplicaciones con calificacion de 5.

% Table generated by Excel2LaTeX from sheet 'Hoja2'
\begin{tabular}{|r|r|r|r|r|}
\hline
  Category &   Installs &        App &     Rating &            \\
\hline
       739 &   BUSINESS &         5+ & EB Cash Collections &        5.0 \\
\hline
       740 &     SOCIAL &        50+ & UP EB Bill Payment & Details &        5.0 \\
\hline
       741 &     SOCIAL &        10+ &    DN Blog &        5.0 \\
\hline
       742 &     SOCIAL &         5+ &  CB Heroes &        5.0 \\
\hline
       743 &       GAME &        50+ & Axe Champs! Wars &        5.0 \\
\hline
       ... &        ... &        ... &        ... &        ... \\
\hline
      1003 &       GAME &        10+ &     211:CK &        5.0 \\
\hline
      1004 &     DATING &       500+ & Spine- The dating app &        5.0 \\
\hline
      1005 &     DATING &       100+ & Online Girls Chat Group &        5.0 \\
\hline
      1006 &      TOOLS &       100+ & BK Formula Calculator &        5.0 \\
\hline
      1007 &      TOOLS &       100+ &  Jabbla BT &        5.0 \\
\hline
\end{tabular}  


- tipos de personas que califican las apps:

\begin{figure} [H]
  \centering
   \includegraphics[width=0.7\linewidth]{personas.png}
  \caption{“tipos de personas que califican las apps.” \cite{alajo2013aplicacion}}
  \label{partes }
\end{figure}

- Apps por genero

\begin{figure} [h]
  \centering
   \includegraphics[width=0.7\linewidth]{GENEROS.png}
  \caption{“Apps por genero.” \cite{alajo2013aplicacion}}
  \label{partes }
\end{figure}



\section{Conclusiones}
Define abbreviations and acronyms the first time they are used in the text, even after they have been defined in the abstract. Abbreviations such as IEEE, SI, MKS, CGS, ac, dc, and rms do not have to be defined. Do not use abbreviations in the title or heads unless they are unavoidable.


\begin{thebibliography}{00}
\bibitem{b1} G. Eason, B. Noble, and I. N. Sneddon, ``On certain integrals of Lipschitz-Hankel type involving products of Bessel functions,'' Phil. Trans. Roy. Soc. London, vol. A247, pp. 529--551, April 1955.
\bibitem{b2} J. Clerk Maxwell, A Treatise on Electricity and Magnetism, 3rd ed., vol. 2. Oxford: Clarendon, 1892, pp.68--73.
\bibitem{b3} I. S. Jacobs and C. P. Bean, ``Fine particles, thin films and exchange anisotropy,'' in Magnetism, vol. III, G. T. Rado and H. Suhl, Eds. New York: Academic, 1963, pp. 271--350.
\bibitem{b4} K. Elissa, ``Title of paper if known,'' unpublished.
\bibitem{b5} R. Nicole, ``Title of paper with only first word capitalized,'' J. Name Stand. Abbrev., in press.
\bibitem{b6} Y. Yorozu, M. Hirano, K. Oka, and Y. Tagawa, ``Electron spectroscopy studies on magneto-optical media and plastic substrate interface,'' IEEE Transl. J. Magn. Japan, vol. 2, pp. 740--741, August 1987 [Digests 9th Annual Conf. Magnetics Japan, p. 301, 1982].
\bibitem{b7} M. Young, The Technical Writer's Handbook. Mill Valley, CA: University Science, 1989.
\end{thebibliography}
\vspace{12pt}
\end{document}
